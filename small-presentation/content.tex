\frame {
   \frametitle{MAlonzo}
   \begin{align*}
      &\mathbf{data}\texttt{ Vec (A : Set) : Nat \(\rightarrow\) Set }\mathbf{where}\\
      &\texttt{\ \ \ nil : Vec A 0}\\
      &\texttt{\ \ \ cons : \{n : Nat\} \(\rightarrow\) A \(\rightarrow\) Vec A n
         \(\rightarrow\) Vec A (n + 1)}\\
      &\texttt{vec-map : \{A B : Set\}\{n : Nat\} \(\rightarrow\) (A \(\rightarrow\) B)
         \(\rightarrow\)}\\
      &\texttt{\ \ \ Vec A n \(\rightarrow\) Vec B n}\\
      &\texttt{vec-map f nil = nil}\\
      &\texttt{vec-map f (cons x xs) = cons (f x) (vec-map f xs)}\\
      \\
      &\mathbf{data}\texttt{ T1 a0 a1 a2 = C2 | C3 a0 a1 a2}\\
      &\texttt{d4 v0 v1 v2 v3 (C2) = cast C2}\\
      &\texttt{d4 v0 v1 v2 v3 (C3 v4 v5 v6) = cast
         (C3\ (cast v4) \dots)}\\
      &\mathbf{ghci}\texttt{>:t d4}\\
      &\texttt{d4 :: a \(\rightarrow\) a1 \(\rightarrow\) a2
         \(\rightarrow\) a3 \(\rightarrow\) (T1 t t1 t2) \(\rightarrow\) b}
   \end{align*}
}

\frame {
   \frametitle{Что хочется}
   \begin{block}{Идея}
   Нужно сохранить инварианты, поддерживаемые системой типов.
   \vspace{1em}
   \begin{enumerate}
   \item Экспортируем весь код в Haskell с помощью MAlonzo.
   \item
      Для выбранных имен пытаемся найти соответствующие типы на Haskell
      и генерируем обертки над скомпилированным кодом.
   \end{enumerate}
   \end{block}
}

\frame {
   \frametitle{Реализованное решение}
   Указание на <<экспорт>> дается с помощью прагм\\
   \texttt{\{-\# EXPORT agda-name haskell-name \#-\}}

   \begin{itemize}
   \item \[\mathbf{data}\texttt{ T (A : Set) : Set \(\rightarrow\) Set }\mathbf{where}\ \ldots\]
      вместе с \texttt{\{-\# EXPORT T Th \#-\}} породит
      \[\mathbf{newtype}\texttt{ Th (a0 :: *) (a1 :: *) = <hidden>}\]
   \item 
      \begin{align*}
         &\texttt{f : \{A : Set\} \(\rightarrow\) (\{B : Set\} \(\rightarrow\) B) \(\rightarrow\) A}\\
         &\texttt{f = \ldots}
      \end{align*}
      вместе с \texttt{\{-\# EXPORT f fh \#-\}} породит
      \begin{align*}
         &\texttt{fh :: \(\forall\)(a0 :: *). (\(\forall\)(a1 :: *). a1) \(\rightarrow\) a0}\\
         &\texttt{fh = \ldots}
      \end{align*}
   \end{itemize}
}

\frame {
   \frametitle{Coq}
   \begin{align*}
      &\mathbf{data}\texttt{ Vec a = Nil | Cons Nat a (Vec a)}\\
      \\
      &\texttt{vecMap :: Nat \(\rightarrow\) (a1 \(\rightarrow\) a2) \(\rightarrow\)
         Vec a1 \(\rightarrow\) Vec a2}\\
      &\texttt{vecMap n f v =}\\
      &\ \ \ \mathbf{case}\ \texttt{v}\ \mathbf{of}\\
      &\texttt{\ \ \ \ Nil \(\rightarrow\) Nil}\\
      &\texttt{\ \ \ \ Cons n0 x xs \(\rightarrow\) Cons n0 (f x)
         (vecMap n0 f xs)}\\
   \end{align*}
}

\frame {
   \frametitle{Agda next}
   \begin{align*}
      &\texttt{f : \{A : Set\} \(\rightarrow\) (\{B : Set\} \(\rightarrow\) B) \(\rightarrow\) A}\\
      &\texttt{f = \ldots}
   \end{align*}
   вместе с \texttt{\{-\# COMPILED\_EXPORT f fh \#-\}} породит
   \begin{align*}
      &\texttt{fh :: \(\forall\)a0. () \(\rightarrow\) (\(\forall\)b0. () \(\rightarrow\) b0) \(\rightarrow\) a0}\\
      &\texttt{fh = \ldots}
   \end{align*}
}

\frame {
   \center{\Huge Q\&A}
}
