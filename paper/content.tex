\section{Введение}

\subsection{Haskell и Agda}

Haskell\footnote{\url{http://haskell.org}} --- функциональный язык программирования
общего назначения.

Agda\footnote{\url{http://wiki.portal.chalmers.se/agda/pmwiki.php?n=Main.HomePage}} ---
функциональный язык программирования с зависимыми типами и, одновременно, ---
система компьютерного доказательства теорем.

\subsection{Экстракция кода}

Термин <<экстракция кода>> пришел из Coq\footnote{\url{http://coq.inria.fr}}
и означает генерацию кода из доказательств\cite{Let02}.

\subsection{Применение экстракции}

\begin{itemize}
\item Техника генерирования верифицированных библиотек --- пишем
      библиотеку с доказательствами на языке с зависимыми типами
      и генерируем код на языке общего назначения вроде Haskell и ML.
      Это позволит использовать написанный код из <<реальных>> приложений,
      который при этом верифицируется системой доказательства теорем.
\item Проще оттранслировать в компилируемый язык, чем написать компилятор.
      А наличие компилятора позволяет улучшить производительность.
\end{itemize}

\newpage
\section{Постановка задачи}

\subsection{Цель}

\subsection{Существующие решения}

\subsection{Анализ MAlonzo}

\subsection{Задачи}

\newpage
\section{Реализация}

\subsection{Архитектура}

\subsection{TODO: ???}

\newpage
\section{Заключение}

\subsection{Выводы}

\subsection{Дальнейшая разработка}
