\section{Введение}

\subsection{Haskell и Agda}

Haskell\footnote{\url{http://haskell.org}} --- функциональный язык программирования
общего назначения.

Agda\footnote{\url{http://wiki.portal.chalmers.se/agda/pmwiki.php?n=Main.HomePage}} ---
функциональный язык программирования с зависимыми типами и, одновременно, ---
система компьютерного доказательства теорем.

\subsection{Экстракция кода}

Термин <<экстракция кода>> пришел из Coq\footnote{\url{http://coq.inria.fr}}
и означает генерацию кода из доказательств\cite{Let02}.

\subsection{Применение экстракции}

\begin{itemize}
\item Техника генерирования верифицированных библиотек --- пишем
      библиотеку с доказательствами на языке с зависимыми типами
      и генерируем код на языке общего назначения вроде Haskell и ML.
      Это позволит использовать написанный код из <<реальных>> приложений,
      который при этом верифицируется системой доказательства теорем.
\item Проще оттранслировать в компилируемый язык, чем написать компилятор.
      А наличие компилятора позволяет улучшить производительность.
\end{itemize}

\newpage
\section{Постановка задачи}

\subsection{Цель}

Разработать способ вызывать код, написанный на Agda, из Haskell.

\subsection{Существующие решения}

\subsubsection{Для Coq}

Coq очень похож на Agda и поэтому имеет смысл сравнивать их технологии.

Техника называется <<экстракция программ>>\cite{Let02}.
По программам на Coq генерируются программы на OCaml, Haskell и Scheme.
При этом все вычисления, выполняемые только на этапе проверки типов, стираются.
То есть, все зависимые типы, и, как следствие, доказательства.

\subsubsection{Для Agda}

На Agda есть компилятор
MAlonzo\footnote{\url{http://thread.gmane.org/gmane.comp.lang.agda/62}}
(являющийся переписанным компилятором Alonzo\cite{Ben07}), который транслирует
код на Agda в код на Haskell и затем компилирует его с помощью ghc, получая в
результате исполняемый файл.

\subsection{Анализ MAlonzo}

TODO: Full description of MAlonzo internals.

\subsection{Задачи}

\begin{enumerate}
\item Реализовать
\item ???
\item PROFIT
\end{enumerate}

TODO: Mention somewhere that the goal is to create a system
TODO: that does not break invariants set up by Agda.

\newpage
\section{Реализация}

\subsection{Архитектура}

\subsection{TODO: ???}

\newpage
\section{Заключение}

\subsection{Выводы}

\subsection{Дальнейшая разработка}
