Ниже приводится принцип трансляции компилятора Alonzo\cite{Ben07}. В самой работе используется
MAlonzo --- переписанный с нуля Alonzo, но сохраняющий правила кодогенерации.

Для трансляции из Agda в Haskell необходимо рассмотреть трансляцию типов данных в типы данных,
типов данных в значения и термов в термы.

\begin{description}
\item[Тип данных в тип данных]
   Несмотря на то, что преобразование теряет информацию о типах, генерировать типы данных необходимо,
   потому что по термам с такими типами выполняется сопоставление с образцом на этапе исполнения.
   Соответственно, каждый конструктор типа данных должен транслироваться в конструктор с такой же
   арностью, чтобы не терять информацию, которая может быть использована при исполнении. Хотя типы
   данных из Agda отвечают GADT из Haskell, генерируются обычные типы данных, теряющие типовые параметры.
   Это гарантирует, что компилятор Haskell на этапе оптимизации не произведет какую-нибудь трансформацию,
   легальную с точки зрения Haskell, но меняющую семантику с точки зрения Agda.
\item[Тип данных в функцию]
   Поскольку в Agda типы и термы могут быть использованы в одном контексте, а в Haskell они раздельны,
   то необходимо для типа сгенерировать функцию, сохраняющую арность типа. То есть, число аргументов
   генерируемой функции должно быть равно числу параметров типа. А поскольку на типах нельзя
   проводить сопоставление с образцом, эта функция может просто возвращать \(()\).
\item[Термы в термы]
   Термы на Agda и на Haskell используют одинаковый синтаксис и одинаковую семантику для всех случаев,
   кроме одного: сопоставление с образцом. Поскольку Agda --- язык с зависимыми типами, то сопоставление
   первого аргумента, определяет тип второго аргумента. Таким образом, можно записать определение функции,
   в которой ведется сопоставление по двум аргументам одновременно, но конструкторы второго аргумента относятся
   к разным типам. В Haskell такое невозможно, поэтому необходимо разбить определение из нескольких клозов на
   несколько определений из одного клоза, вызывающих друг друга по цепочке, и использовать \texttt{unsafeCoerce}
   для второго аргумента. Еще одно отличие, вызванное сопоставлением с образцом --- наличие невозможного образца
   в Agda. Поскольку тип второго аргумента может меняться, в каком-то клозе он может стать типом, у которого нет
   конструктоов и в Agda есть специальный синтаксис, чтобы пометить этот аргумент в таком клозе и не писать
   реализацию этого клоза. В Haskell такого не существует и поэтому код этого клоза заменяется на вызов функции
   \texttt{error}.
\end{description}
