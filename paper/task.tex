\section{Постановка задачи}

\subsection{Цель}\label{sec:task-goal}

Разработать способ вызывать код, написанный на Agda, из Haskell, не нарушая
внутренних инвариантов, установленных Agda.

Про сохранение внутренних инвариантов нужно объяснить подробнее. Рассмотрим пример:
\begin{align*}
&\mathbf{data}\ Nat : Set\ \mathbf{where}\\
&\quad zero : Nat\\
&\quad succ : Nat \rightarrow Nat\\
\\
&\mathbf{data}\ List\ (A : Set) : Set\ \mathbf{where}\\
&\quad nil : List\ A\\
&\quad cons : A \rightarrow List\ A \rightarrow List\ A\\
\\
&length : \forall\ \{A\} \rightarrow List\ A \rightarrow Nat\\
&length\ nil = zero\\
&length\ (cons\ \_\ xs) = succ\ (length\ xs)\\
\\
&\mathbf{data}\ Fin : Nat \rightarrow Set\ \mathbf{where}\\
&\quad finzero : \forall\ \{n\} \rightarrow Fin\ (succ\ n)\\
&\quad finsucc : \forall\ \{n\} \rightarrow Fin\ n \rightarrow Fin\ (succ\ n)\\
\\
&elemAt : \forall\ \{A\}\ (xs : List\ A) \rightarrow Fin\ (length\ xs) \rightarrow A\\
&elemAt\ nil\ ()\\
&elemAt\ (cons\ x\ \_)\ finzero = x\\
&elemAt\ (cons\ \_\ xs)\ (finsucc\ n) = elemAt\ xs\ n
\end{align*}

В этом коде определяются три типа данных: натуральныe числа,
список и конечные числа (тип \(Fin\ n\) имеют числа, меньшие \(n\)) и
2 функции: длина списка и получение элемента из списка по индексу.
Рассмотрим вторую функцию. Она принимает 2 аргумента: список и число,
меньшее длины списка. Это гарантирует, что элемент с таким индексом
существует. Этот инвариант используется в самом первом клозе:
при попытке написать код для пустого списка, Agda замечает, что нет способа
построить терм с типом \(Fin\ zero\) и поэтому вместо тела пишется \(()\).
А так как система типов гарантирует, что этот клоз не будет вызван, то
никакой ошибки на этапе исполнения быть не может.

При экстракции в Haskell хочется сохранить это свойство. Но \(elemAt\)
использует зависимые типы, которые не получится воспроизвести на Haskell.
Код, сгенерированный из Agda, будет поддерживать это свойство, но для
внешнего кода дать гарантий не получится. Поэтому, необходимо запретить
вызов этой функции.

\subsection{Существующие решения}

\subsubsection{Для Coq}

Как было сказано в пункте \ref{sec:intro-extraction} в Coq есть технология
<<экстракции программ>>. Но текущая реализация стирает все зависимые типы и
код аналогичный \ref{sec:task-goal}: (TODO: I'm having trouble implementing elemAt in Coq)

\begin{align*}
&\mathbf{Inductive}\ Nat : Set :=\\
&\ |\ zero : Nat\\
&\ |\ succ : Nat \rightarrow Nat.\\
\\
&\mathbf{Inductive}\ List\ (A : Type) : Type :=\\
&\ |\ nil : List\ A\\
&\ |\ cons : A \rightarrow List\ A \rightarrow List\ A.\\
\\
&\mathbf{Fixpoint}\ length\ (A : Type)\ (xs : List\ A)\ \{struct\ xs\} : Nat :=\\
&\quad\mathbf{match}\ xs\ \mathbf{with}\\
&\quad\ |\ nil \Rightarrow zero\\
&\quad\ |\ cons\ \_\ xs' \Rightarrow succ\ (length\ \_\ xs')\\
&\quad\mathbf{end}.\\
\\
&\mathbf{Inductive}\ Fin : Nat \rightarrow Set :=\\
&\ |\ finzero : \mathbf{forall}\ n : Nat,\ Fin\ (succ\ n)\\
&\ |\ finsucc : \mathbf{forall}\ n : Nat,\ Fin\ n \rightarrow Fin\ (succ\ n).\\
\\
&\mathbf{Lemma}\ emptyfin : \mathbf{forall}\ f : Fin\ zero,\ False.\\
&\mathbf{Proof}.\ intros\ H;\ inversion\ H.\ \mathbf{Qed}.\\
\\
&\mathbf{Fixpoint}\ elemAt\ (A : Type)\ (xs : List\ A)\ (n : Fin\ (length\ \_\ xs)) : A :=\\
&\quad\mathbf{match}\ xs,\ n\ \mathbf{with}\\
&\quad\ |\ nil,\ \_ \Rightarrow \mathbf{match}\ emptyfin\ n\ \mathbf{with\ end}\\
&\quad\ |\ cons\ x\ \_,\ finzero\ \_ \Rightarrow x\\
&\quad\ |\ cons\ \_\ xs',\ finsucc\ \_\ n' \Rightarrow elemAt\ \_\ xs'\ n'
\ (*\text{ TODO: This case fails }*)
\end{align*}

будет преобразован в:
\begin{align*}
&\mathbf{data}\ Nat = Zero\ |\ Succ\ Nat\\
\\
&\mathbf{data}\ List\ a = Nil\ |\ Cons\ a\ (List\ a)\\
\\
&length :: List\ a1 \rightarrow Nat\\
&length\ Nil = Zero\\
&length\ (Cons\ a\ xs') = Succ\ (length\ xs')\\
\\
&\mathbf{data}\ Fin = Finzero\ Nat\ |\ Finsucc\ Nat\ Fin\\
\\
&\texttt{--- TODO: Was not able to implement correctly in Coq}\\
&elemAt :: List\ a1 \rightarrow Fin \rightarrow a1\\
&elemAt\ Nil\ n = \texttt{error}\ "absurd\ case"\\
&elemAt\ (Cons\ x\ \_)\ (Finzero\ \_) = x\\
&elemAt\ (Cons\ \_\ xs')\ (Finsucc\ \_\ n') = elemAt\ xs'\ n'
\end{align*}

И теперь можно вызвать \(elemAt\) от пустого списка и получить ошибку на
этапе выполнения, что нежелательно.

\subsubsection{Для Agda}

На Agda есть компилятор
MAlonzo\footnote{\url{http://thread.gmane.org/gmane.comp.lang.agda/62}}
(являющийся переписанным компилятором Alonzo\cite{Ben07}), который транслирует
код на Agda в код на Haskell и затем компилирует его с помощью
ghc\footnote{\url{http://www.haskell.org/ghc/}} (де-факто стандарт Haskell),
получая в результате исполняемый файл.

Из-за фокуса на генерацию исполняемых файлов, а не библиотек,
генерируемый код создает имена вида буква+число, для функций
не выписывается их тип и типы данных теряют типовые параметры.
Пример из \ref{sec:task-goal} преобразуется приблизительно в:

\begin{align*}
&name1 = "Main.Nat"\\
&name2 = "Main.Nat.zero"\\
&name3 = "Main.Nat.zero"\\
&d1 = ()\\
&\mathbf{data}\ T1\ a0 = C2\ |\ C3\ a0\\
\\
&name5 = "Main.List"\\
&name7 = "Main.List.nil"\\
&name8 = "Main.List.cons"\\
&d5\ a0 = ()\\
&\mathbf{data}\ T5\ a0\ a1 = C7\ |\ C8\ a0\ a1\\
\\
&name10 = "Main.length"\\
&d10\ v0\ C7 = \texttt{unsafeCoerce}\ C2\\
&d10\ v0\ (C8\ v1\ v2) = \texttt{unsafeCoerce}\ (C3\ (\texttt{unsafeCoerce}\\
&\quad(d10\ (\texttt{unsafeCoerce}\ v0)\ (\texttt{unsafeCoerce}\ v1))))\\
\\
&name12 = "Main.Fin"\\
&name14 = "Main.Fin.finzero"\\
&name16 = "Main.Fin.finsucc"\\
&d12\ a0 = ()\\
&\mathbf{data}\ T12\ a0\ a1 = C14\ a0\ |\ C16\ a0\ a1\\
\\
&name19 = "Main.elemAt"\\
&d19\ \_\ C7\ \_ = \texttt{error}\ "Impossible\ clause\ body"\\
&d19\ v0\ (C8\ v1\ v2)\ (C14\ v3) = \texttt{unsafeCoerce}\ v1\\
&d19\ v0\ (C8\ v1\ v2)\ (C16\ v3\ v4) = \texttt{unsafeCoerce}\ (d19\\
&\quad(\texttt{unsafeCoerce}\ v0)\ (\texttt{unsafeCoerce}\ v2)\ (
   \texttt{unsafeCoerce}\ v4))
\end{align*}

Как видно, это делает использование сгенерированного кода
практически неприменимым.

\paragraph{Замечание об Agda 2.3.4}

Работа начиналась на версии 2.3.2.2, в версии 2.3.4 появились 2 релевантные возможности:
\begin{itemize}
\item давать функциям пользовательские имена и заставлять MAlonzo генерировать для них
      более-менее осмысленные типы с помощью прагмы
      \texttt{\{-\# COMPILED\_EXPORT \(AgdaName\) \(HaskellName\) \#-\}}
\item флаг \texttt{---compile-no-main}, который не требует наличие функции \(main\).
\end{itemize}

\subsection{Анализ MAlonzo}

TODO: Full description of MAlonzo internals.

\subsection{Задачи}

\begin{enumerate}
\item Придумать способ генерировать интерфейс на Haskell, не позволяющий нарушить инварианты,
      поддерживаемые Agda
\item Реализовать
\item Доказать корректность способа
\end{enumerate}
