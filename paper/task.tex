\section{Постановка задачи}

\subsection{Цель}\label{sec:task-goal}

Разработать способ вызывать код, написанный на Agda, из Haskell, не нарушая
внутренних инвариантов, установленных Agda.

\subsection{Сохранение внутренних инвариантов}

\subsubsection{Скрытие зависимых типов}

Рассмотрим пример:

\label{text:task-agda-code}
\begin{align*}
&\mathbf{module}\ Main\ \mathbf{where}\\
\\
&\mathbf{data}\ Nat : Set\ \mathbf{where}\\
&\quad zero : Nat\\
&\quad succ : Nat \rightarrow Nat\\
\\
&\mathbf{data}\ List\ (A : Set) : Set\ \mathbf{where}\\
&\quad nil : List\ A\\
&\quad cons : A \rightarrow List\ A \rightarrow List\ A\\
\\
&length : \forall\ \{A\} \rightarrow List\ A \rightarrow Nat\\
&length\ nil = zero\\
&length\ (cons\ \_\ xs) = succ\ (length\ xs)\\
\\
&\mathbf{data}\ Fin : Nat \rightarrow Set\ \mathbf{where}\\
&\quad finzero : \forall\ \{n\} \rightarrow Fin\ (succ\ n)\\
&\quad finsucc : \forall\ \{n\} \rightarrow Fin\ n \rightarrow Fin\ (succ\ n)\\
\\
&elemAt : \forall\ \{A\}\ (xs : List\ A) \rightarrow Fin\ (length\ xs) \rightarrow A\\
&elemAt\ nil\ ()\\
&elemAt\ (cons\ x\ \_)\ finzero = x\\
&elemAt\ (cons\ \_\ xs)\ (finsucc\ n) = elemAt\ xs\ n
\end{align*}

В этом коде определяются три типа данных: натуральныe числа,
список и конечные числа (тип \(Fin\ n\) имеют числа, меньшие \(n\)) и
2 функции: длина списка и получение элемента из списка по индексу.
Рассмотрим вторую функцию. Она принимает 3 аргумента: тип элементов
в списке, список и число, меньшее длины списка. Таким образом, есть
гарантия, что эта функция всегда будет вызвана с индексом внутри списка.
Этот инвариант используется в самом первом клозе:
при попытке написать код для пустого списка, Agda замечает, что нет способа
построить терм с типом \(Fin\ zero\) и поэтому вместо тела пишется \(()\).
А так как система типов гарантирует, что этот клоз не будет вызван, то
никакой ошибки на этапе исполнения быть не может.

\label{text:limited-interface}
При экстракции в Haskell хочется сохранить это свойство. Но \(elemAt\)
использует зависимые типы, которые не получится воспроизвести на Haskell.
Код, сгенерированный из Agda, будет поддерживать это свойство, но для
внешнего кода дать гарантий не получится. Поэтому, необходимо запретить
вызов этой функции извне.

\subsubsection{Скрытие конструкторов типов данных}\label{sec:task-hiding-constructors}

Рассмотрим пример:

\begin{align*}
&\mathbf{data}\ \bot\ \mathbf{where}\\
\\
&\mathbf{data}\ \_\equiv\_\ \{A : Set\}\ (x : A) : A \rightarrow Set\ \mathbf{where}\\
&\quad refl : x \equiv x\\
\\
&\_\not\equiv\_ : \{A : Set\} \rightarrow A \rightarrow A \rightarrow Set\\
&x \not\equiv y = x \equiv y \rightarrow \bot\\
\\
&\mathbf{data}\ IsEqual\ (A : Set) : Set\ \mathbf{where}\\
&\quad yes : (x\ y : A) \rightarrow x \equiv y \rightarrow IsEqual\ A\\
&\quad no : (x\ y : A) \rightarrow x \not\equiv y \rightarrow IsEqual\ A\\
\\
&isEqualNat : Nat \rightarrow Nat \rightarrow IsEqual\ Nat\\
&isEqualNat = \cdots\\
\\
&someFunction : IsEqual\ Nat \rightarrow SomeType\\
&someFunction = \cdots
\end{align*}

В примере задаются 4 типа: пустой тип, равенство, неравенство
(как отрицание равенства) и тип проверки на равенство. У последнего
в конструкторах содержатся соответствующие доказательства. Из-за этого
невозможно полностью этот тип представить в Haskell. Но если сделать этот
тип абстрактным (скрыть конструкторы), то его станет можно использовать из
Haskell: построение с помощью \(isEqualNat\) и использование с помощью
\(someFunction\).

\subsubsection{Обработка простых типов}

С другой стороны, такие простые типы как \(Nat\), \(List\) и
функция \(length\) могут быть напрямую представлены в Haskell:

\begin{align*}
&\mathbf{data}\ Nat = Zero\ |\ Succ\ Nat\\
\\
&\mathbf{data}\ List\ a = Nil\ |\ Cons\ a\ (List\ a)\\
\\
&length :: List\ a \rightarrow Nat\\
&length\ Nil = Zero\\
&length\ (Cons\ \_\ xs) = Succ\ (length\ xs)
\end{align*}

Несмотря на это, типы \(Nat\) и \(List\) все равно будут выставлены
как абстрактные типы. Причина раскрывается в \ref{sec:implementation-data-trouble}.

\subsection{Существующие решения}

\subsubsection{Для Coq}

Как было сказано в пункте \ref{sec:intro-extraction} в Coq есть технология
<<экстракции программ>>. Но текущая реализация стирает все зависимые типы и
код аналогичный \ref{text:task-agda-code}:

\begin{align*}
&\mathbf{Inductive}\ Nat : Set :=
   |\ zero : Nat\ |\ succ : Nat \rightarrow Nat.\\
&\mathbf{Conjecture}\ succ\_zero : \mathbf{forall}\ n,\ succ\ n = zero \rightarrow
   False.\\
&\mathbf{Definition}\ succ\_succ\ \{n1\}\ \{n2\}\ (e : succ\ n1 = succ\ n2) :
   n1 = n2 :=\\
&\quad\mathbf{match}\ e\ \mathbf{with}\ |\ eq\_refl \Rightarrow
   eq\_refl\ \mathbf{end}.\\
\\
&\mathbf{Inductive}\ List\ (A : Type) : Type :=\\
&|\ nil : List\ A\\
&|\ cons : A \rightarrow List\ A \rightarrow List\ A.\\
&\mathbf{Fixpoint}\ length\ \{A\}\ (xs : List\ A)\ \{struct\ xs\} : Nat :=\\
&\quad\mathbf{match}\ xs\ \mathbf{with}\\
&\quad|\ nil \Rightarrow zero\\
&\quad|\ cons\ \_\ xs' \Rightarrow succ\ (length\ xs')\\
&\quad\mathbf{end}.\\
\\
&\mathbf{Inductive}\ Fin : Nat \rightarrow Set :=\\
&|\ finzero : \mathbf{forall}\ n : Nat,\ Fin\ (succ\ n)\\
&|\ finsucc : \mathbf{forall}\ n : Nat,\ Fin\ n \rightarrow Fin\ (succ\ n).\\
&\mathbf{Definition}\ finofzero\ \{n\}\ (f : Fin\ n) : n = zero \rightarrow False :=\\
&\quad\mathbf{match}\ f\ \mathbf{with}\\
&\quad|\ finzero\ \_ \Rightarrow \mathbf{fun}\ e \Rightarrow succ\_zero\ \_\ e\\
&\quad|\ finsucc\ \_\ \_ \Rightarrow \mathbf{fun}\ e \Rightarrow succ\_zero\ \_\ e\\
&\quad\mathbf{end}.\\
\\
&\mathbf{Fixpoint}\ elemAt'\ \{A\}\ \{n\}\ (xs : List\ A)\ (i : Fin\ n) :
   n = length\ xs \rightarrow A :=\\
&\quad\mathbf{match}\ xs,\ i\ \mathbf{with}\\
&\quad|\ nil,\ \_ \Rightarrow \mathbf{fun}\ (e : n = length\ (nil\ \_))
   \Rightarrow \mathbf{match}\ finofzero\ i\ e\ \mathbf{with\ end}\\
&\quad|\ cons\ x\ \_,\ finzero\ \_ \Rightarrow \mathbf{fun}\ \_ \Rightarrow x\\
&\quad|\ cons\ \_\ xs',\ finsucc\ \_\ i' \Rightarrow \mathbf{fun}\ e \Rightarrow
   elemAt'\ xs'\ i'\ (succ\_succ\ e)\\
&\quad\mathbf{end}.\\
\\
&\mathbf{Fixpoint}\ elemAt\ \{A\}\ (xs : List\ A)\ (i : Fin\ (length\ xs)) : A :=\\
&\quad elemAt'\ xs\ i\ eq\_refl.
\end{align*}

будет преобразован примерно в:
\begin{align*}
&\mathbf{data}\ Nat = Zero\ |\ Succ\ Nat\\
\\
&\mathbf{data}\ List\ a = Nil\ |\ Cons\ a\ (List\ a)\\
\\
&length :: List\ a1 \rightarrow Nat\\
&length\ Nil = Zero\\
&length\ (Cons\ a\ xs') = Succ\ (length\ xs')\\
\\
&\mathbf{data}\ Fin = Finzero\ Nat\ |\ Finsucc\ Nat\ Fin\\
\\
&elemAt' :: Nat \rightarrow List\ a1 \rightarrow Fin \rightarrow a1\\
&elemAt'\ \_\ Nil\ \_ = \texttt{error}\ "absurd\ case"\\
&elemAt'\ \_\ (Cons\ x\ \_)\ (Finzero\ \_) = x\\
&elemAt'\ \_\ (Cons\ \_\ xs')\ (Finsucc\ n0\ i') = elemAt'\ n0\ xs'\ i'\\
\\
&elemAt :: (List\ a1) \rightarrow Fin \rightarrow a1\\
&elemAt\ xs\ i = elemAt'\ (length\ xs)\ xs\ i
\end{align*}

И теперь можно вызвать \(elemAt\) от пустого списка и получить ошибку на
этапе выполнения, что нежелательно.

\subsubsection{Для Agda}

На Agda есть компилятор
MAlonzo\footnote{\url{http://thread.gmane.org/gmane.comp.lang.agda/62}}
(являющийся переписанным компилятором Alonzo\cite{Ben07}), который транслирует
код на Agda в код на Haskell и затем компилирует его с помощью
ghc\footnote{\url{http://www.haskell.org/ghc/}} (де-факто стандарт Haskell),
получая в результате исполняемый файл.

Из-за фокуса на генерацию исполняемых файлов, а не библиотек,
генерируемый код создает имена вида буква+число, для функций
не выписывается их тип и типы данных теряют типовые параметры.
Пример из \ref{text:task-agda-code} преобразуется приблизительно в:

\begin{align*}
&name1 = "Main.Nat"\\
&name2 = "Main.Nat.zero"\\
&name3 = "Main.Nat.zero"\\
&d1 = ()\\
&\mathbf{data}\ T1\ a0 = C2\ |\ C3\ a0\\
\\
&name5 = "Main.List"\\
&name7 = "Main.List.nil"\\
&name8 = "Main.List.cons"\\
&d5\ a0 = ()\\
&\mathbf{data}\ T5\ a0\ a1 = C7\ |\ C8\ a0\ a1\\
\\
&name10 = "Main.length"\\
&d10\ v0\ C7 = \texttt{unsafeCoerce}\ C2\\
&d10\ v0\ (C8\ v1\ v2) = \texttt{unsafeCoerce}\ (C3\ (\texttt{unsafeCoerce}\\
&\quad(d10\ (\texttt{unsafeCoerce}\ v0)\ (\texttt{unsafeCoerce}\ v1))))\\
\\
&name12 = "Main.Fin"\\
&name14 = "Main.Fin.finzero"\\
&name16 = "Main.Fin.finsucc"\\
&d12\ a0 = ()\\
&\mathbf{data}\ T12\ a0\ a1 = C14\ a0\ |\ C16\ a0\ a1\\
\\
&name19 = "Main.elemAt"\\
&d19\ \_\ C7\ \_ = \texttt{error}\ "Impossible\ clause\ body"\\
&d19\ v0\ (C8\ v1\ v2)\ (C14\ v3) = \texttt{unsafeCoerce}\ v1\\
&d19\ v0\ (C8\ v1\ v2)\ (C16\ v3\ v4) = \texttt{unsafeCoerce}\ (d19\\
&\quad(\texttt{unsafeCoerce}\ v0)\ (\texttt{unsafeCoerce}\ v2)\ (
   \texttt{unsafeCoerce}\ v4))
\end{align*}

Параметры для типов данных здесь используются только, чтобы объявить
их для использования в конструкторах: \(T12\), к примеру, мог бы аналогично
(с точки зрения сохранения информации о типах) объявляться как:

\begin{align*}
&\mathbf{data}\ T12\ \mathbf{where}\\
&\quad C14 :: a0 \rightarrow T12\\
&\quad C16 :: a0 \rightarrow a1 \rightarrow T12
\end{align*}

Как видно, это делает использование сгенерированного кода
практически неприменимым.

\paragraph{Замечание об Agda 2.3.4}

Работа начиналась на версии 2.3.2.2, в версии 2.3.4 появились 2 релевантные возможности:
\begin{itemize}
\item давать функциям пользовательские имена и заставлять MAlonzo генерировать для них
      осмысленные типы с помощью прагмы
      \texttt{\{-\# COMPILED\_EXPORT \(AgdaName\) \(HaskellName\) \#-\}}
      \label{text:agda-compiled-export}
\item флаг \texttt{---compile-no-main}, который позволяет компилировать код в
      библиотеку, а не исполняемый файл.
\end{itemize}

Генерируемый тип для функций выполняется по следующим правилам\footnote{\mbox{\url
{http://wiki.portal.chalmers.se/agda/pmwiki.php?n=ReferenceManual.ForeignFunctionInterface
}}}:

\begin{align*}
T\llbracket Set \rrbracket &= ()\\
T\llbracket x\ As \rrbracket &= x\ T\llbracket As \rrbracket\\
T\llbracket \lambda (x : A) \rightarrow B \rrbracket &= undef\\
T\llbracket (x : A) \rightarrow B \rrbracket &= \left\{
   \begin{array}{l l}
      \mathbf{forall}\ x.\ T\llbracket A \rrbracket \rightarrow T\llbracket B \rrbracket
      &\quad x \in fv\ B\\
      T\llbracket A \rrbracket \rightarrow T\llbracket B \rrbracket
      &\quad \text{иначе}
   \end{array}\right.\\
T\llbracket k\ As \rrbracket &= \left\{
   \begin{array}{l l}
      T\ T\llbracket As \rrbracket &\quad \texttt{COMPILED\_TYPE}\ k\ T\\
      () &\quad \texttt{COMPILED}\ k\ E\\
      undef &\quad \text{иначе}
   \end{array}\right.
\end{align*}

\subsection{Задачи}

\begin{enumerate}
\item Разобраться со способом генерации кода MAlonzo.
\item Придумать способ генерировать интерфейс на Haskell к сгенерированному
      MAlonzo коду, не позволяющий нарушить инварианты, поддерживаемые Agda.
\item Доказать корректность генерируемого интерфейса.
\item Реализовать поддержку этого механизма в компиляторе MAlonzo и протестировать.
\end{enumerate}
