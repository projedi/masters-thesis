\section{Введение}

\subsection{Haskell и Agda}

Haskell\footnote{\url{http://haskell.org}} --- функциональный язык программирования
общего назначения.

Agda\footnote{\url{http://wiki.portal.chalmers.se/agda/pmwiki.php?n=Main.HomePage}} ---
функциональный язык программирования с зависимыми типами и, одновременно, ---
система компьютерного доказательства теорем.

\subsection{Зависимые типы}

\subsubsection{Определение зависимых типов}

Зависимый тип --- тип, зависящий от значения. К примеру,
\[
vecSum : \{n : Nat\} \rightarrow Vec\ Nat\ n \rightarrow Vec\ Nat\ n \rightarrow Vec\ Nat\ n.
\]
Эта функция берет 2 вектора и делает попарную сумму.
Вектор --- это односвязный список, в типе которого записана его длина.
Таким образом, функция принимает 3 аргумента\footnote{
Первый аргумент в фигурных скобках --- синтаксический сахар, позволяющий программисту
не писать этот аргумент при вызове функции, так как система может его вывести самостоятельно.}:
число \(n\), 2 вектора длины \(n\) --- и возвращает вектор длины \(n\). Тип этой функции зависимый,
потому что если на первый аргумент передать \(0\), то ее тип будет
\(Nat \rightarrow Vec\ Nat\ 0 \rightarrow Vec\ Nat\ 0 \rightarrow Vec\ Nat\ 0\), а если передать \(3\), то ---
\(Nat \rightarrow Vec\ Nat\ 3 \rightarrow Vec\ Nat\ 3 \rightarrow Vec\ Nat\ 3\).

Рассмотрим еще один пример:
\[
vecFromListNat : (xs : List\ Nat) \rightarrow Vec\ Nat\ (length\ xs).
\]
Эта функция строит из обычного односвязного списка вектор. В ее типе встроен вызов
функции \(length\). И если вызвать ее от пустого списка, то тип будет
\(List\ Nat \rightarrow Vec\ Nat\ 0\), а если от списка из \(5\) элементов, то ---
\(List\ Nat \rightarrow Vec\ Nat\ 5\).

Возможность смешивать типы и термы приводит к тому, что эти 2 множества неразличимы
(в отличие, к примеру, от System F). Таким образом, необходима возможность давать <<типы>>
для типов. Эту роль исполняет \(Set_0\). К примеру, верно, что \(List\ Nat : Set_0\). В свою
очередь, \(Set_0 : Set_1\) и так далее.

\subsubsection{Связь с логикой}

Соответствие Карри-Говарда --- связь между программами и интуиционистской логикой\footnote{
Еще известна как конструктивная логика}. В частности, утверждается, что типы соответствуют
логическим утверждениям, а термы --- доказательствам, в том смысле, что если существует
терм с каким-то типом, то логическое утверждение, соответствующее этому типу будет тавтологией.

Зависимым типам соответствует логика первого порядка:
\begin{description}
\item[\(A \rightarrow B\)] соотвествует импликации,
\item[\((x : T_1) \rightarrow T_2\)] соответствует кванторy всеобщности,
\item[\(T_1 : T_2 \rightarrow Set\)] соответствует предикату (например, вместо \(T_1\) можно подставить \(Vec\ Nat\),
   а вместо \(T_2\) --- \(Nat\))
\item[\(\bot\)] соответствует \texttt{FALSE} (\(\bot\) --- тип данных без единого конструктора).
\end{description}

\subsection{Экстракция кода}\label{sec:intro-extraction}

Термин <<экстракция программ>> пришел из языка/cистемы доказательства теорем
Coq\footnote{\url{http://coq.inria.fr}}, похожего на Agda, и означает генерацию
функционального кода из доказательств \cite{Let02}.

\subsection{Применение экстракции}

Можно выделить 2 основных причины для реализации механизма экстракции:

\paragraph{Бесплатная компилируемость}

Скомпилированный код как правило работает быстрее интерпретации, а умение
транслировать код в компилируемый язык освобождает от сложной задачи
написания компилятора с нуля.

\paragraph{Генерирование верифицированных библиотек}

На системах с зависимыми типами вроде Agda и Coq можно строить
сложные логические утверждения, которые будут проверяться на этапе
проверки типов (за счет чего эти системы помогают формально доказывать теоремы).
Таким образом, можно написать библиотеку на таком языке с набором доказанных
свойств и после этого сделать экстракцию в язык вроде Haskell или ML,
на которых проще писать <<реальные>> программы. В этой работе фокус ставится на этот пункт.
