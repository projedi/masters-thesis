\section{Введение}

\subsection{Haskell и Agda}

Haskell\footnote{\url{http://haskell.org}} --- функциональный язык программирования
общего назначения.

Agda\footnote{\url{http://wiki.portal.chalmers.se/agda/pmwiki.php?n=Main.HomePage}} ---
функциональный язык программирования с зависимыми типами и, одновременно, ---
система компьютерного доказательства теорем.

\subsection{TODO: Зависимые типы?}

\subsection{Экстракция кода}\label{sec:intro-extraction}

Термин <<экстракция программ>> пришел из языка/cистемы доказательства теорем
Coq\footnote{\url{http://coq.inria.fr}}, похожего на Agda, и означает генерацию
функционального кода из доказательств \cite{Let02}.

\subsection{Применение экстракции}

Можно выделить 2 основных причины для реализации механизма экстракции:
\begin{enumerate}
\item \textbf{Техника генерирования верифицированных библиотек}

      На системах с зависимыми типами вроде Agda и Coq можно строить
      сложные логические утверждения, которые будут проверяться на этапе
      проверки типов (за счет чего эти системы помогают формально доказывать теоремы).
      Таким образом, можно написать библиотеку на таком языке с набором доказанных
      свойств и после этого сделать экстракцию в язык вроде Haskell или ML,
      на которых проще писать <<реальные>> программы.

\item \textbf{Бесплатная компилируемость}

      Скомпилированный код как правило работает быстрее интерпретации, а умение
      транслировать код в компилируемый язык освобождает от сложной задачи
      написания компилятора с нуля.
\end{enumerate}

В этой работе фокус ставится на первый пункт.
