\section{Реализация}

\subsection{Анализ MAlonzo}

TODO: Full description of MAlonzo internals.

\subsection{Генерирование ограниченного интерфейса}

Как обсуждалось в \ref{text:limited-interface}, необходим способ ограничивать
функционал генерируемого интерфейса, чтобы выставлять только те элементы,
использование которых не может нарушить внутренние инварианты системы.

Также необходимо уметь генерировать имена для получаемого интерфейса:
правила именования в Agda и Haskell отличаются.

Поэтому, было решено ввести прагму
\[
\texttt{\{-\# EXPORT \(AgdaName\) \(HaskellName\) \#-\}},
\]
которая пишется в том же модуле \(AgdaModuleName\), где определяется \\\(AgdaName\).

Если cущность \(AgdaName\) не может быть сконвертирована в аналогичную на Haskell
или \(HaskellName\) не соблюдает правила именования, то компиляция завершится с ошибкой.

На этапе компиляции вместе с \texttt{MAlonzo.Code.\(AgdaModuleName\)}, где хранится
код генерируемый MAlonzo, будет сгенерирован \\\texttt{MAlonzo.Export.\(AgdaModuleName\)},
в котором находится весь генерируемый интерфейс.

Решение создать отдельный модуль \texttt{MAlonzo.Export} вызвано желанием скрыть
сгенерированный MAlonzo код от пользователя. В том числе это позволит
при генерировании документации с помощью
haddock\footnote{\url{http://www.haskell.org/haddock/}} отображать только желаемый
интерфейс.

\subsection{Встраивание в MAlonzo}

Вместо изменения кода, который генерирует MAlonzo было решено генерировать
обертки, имеющие нужный интерфейс и вызывающие код, сгенерированный MAlonzo.
Это позволяет менять меньше кода в MAlonzo, хотя может повлиять на производительность:
оборачивание функций может быть не отброшено оптимизатором.

Кодогенерация вызывается на следующих участках:
\begin{enumerate}
\item При начале обработки модуля \(AgdaModuleName\) компилятором MAlonzo,
      контекст, содержащий сгенерированный код, обнуляется.
\item После обработки каждого определения \(AgdaName\): функции или типа данных ---
      проверяется наличие прагмы \\\texttt{\{-\# EXPORT \(AgdaName\) \(HaskellName\) \#-\}}.
      Если она не найдена, это определение пропускается; иначе - выполняется проверка
      на возможность сгенерировать интерфейс на Haskell. При неудаче выдается ошибка,
      иначе - в контекст добавляется сгенерированные обертки.
\item После обработки модуля, если контекст не пуст, создается модуль\\
      \texttt{MAlonzo.Export.\(AgdaModuleName\)}, в который записывается код из контекста.
\end{enumerate}

\subsection{Выполняемая кодогенерация}

Формально задается в приложении \ref{sec:appendix-transformations}.
Корректность доказывается в приложении \ref{sec:appendix-proof}.

\subsubsection{Типы данных}\label{sec:implementation-datatypes}

В \ref{sec:task-hiding-constructors} говорилось, что некоторые типы данных
(вводимые в Agda с помощью \textbf{data} и \textbf{record}) можно
представить в Haskell, только если cделать их абстрактными.

Конкретнее, если объявление типа имеет вид:
\[
\mathbf{data}\ AgdaName\ (A_0 : S_0) \cdots (A_m : S_m) :
S_{m+1} \rightarrow \cdots \rightarrow S_n \rightarrow Set_0\ \mathbf{where}\ \dots,
\]
где \(S_i\) - комбинация \(Set_0\)
и \(\rightarrow\), то аналогичным объявлением на Haskell будет:

\[
\mathbf{newtype}\ HaskellName\ (a_0 :: K_0) \cdots (a_n :: K_n) = \dots,
\]
где \(K_i\) --- \(S_i\), в котором \(Set_0\) заменяется на \(*\).
Введенное ограничение на \(S_i\) гарантирует, что такой тип представим
в Haskell: структурная индукция по \(K\)
\begin{description}
\item[\(*\): ]
\[
\mathbf{data}\ T :: *
\]
\item[\(K_0 \rightarrow \cdots \rightarrow K_n \rightarrow *\): ]
\[
\mathbf{data}\ T\ (a_0 :: K_0) \ldots (a_n :: K_n) :: *.
\]
\(K_i\) представимы по индукции.
\end{description}

\(Set_0\) --- консервативный выбор. Существуют типы из \(Set_1\), представимые в Haskell
с помощью экзистенциальных типов. Но было принято решение ограничиться \(Set_0\),
потому что в Agda верно \(Set_0 : Set_1\), а в Haskell \(* :: *\) --- неверно.

Теперь, MAlonzo по \(AgdaName\) сгенерирует
\[
\mathbf{data}\ T_k\ a_0 \cdots a_p = \dots
\]
и если сгенерировать
\begin{align*}
&\mathbf{newtype}\ HaskellName\ (a_0 :: K_0) \cdots (a_n :: K_n) = \\
&\quad HaskellName\ (\mathbf{forall}\ b_0 \cdots b_p .\ T_k\ b_0 \cdots b_p),
\end{align*}
то для трансформации между термом из MAlonzo, имеющим тип \(T_k\ args\dots\) и
термом, выставленным наружу, имеющим тип \(HaskellName\ args\dots\)
можно использовать \texttt{unsafeCoerce}, так как \textbf{newtype} гарантирует\cite{HaskellNewtype}
идентичное внутреннее представление.

\subsubsection{Типы функций}\label{sec:implementation-functions}

Типы функций \(T\), представимые в Haskell, имеют следующий вид:
\begin{align*}
&(A : S) \rightarrow T,\quad\text{\(S\) состоит из \(Set_0\) и \(\rightarrow\)}\\
&X\ args\dots,\quad\text{\(args\) представимы и \(X\) --- типовой параметр,
тип из \ref{sec:implementation-datatypes},}\\
&\qquad\text{FFI-импортированный тип или встроенный тип}\\
&(x : T_1) \rightarrow T_2,\quad x \not\in freevars(T_2)\\
\end{align*}

\paragraph{Тип зависимой функции с неиспользуемыми аргументами}
На Haskell будут иметь вид
\begin{align*}
&T_1 \rightarrow T_2\\
\end{align*}
то есть, абсолютно эквивалентен оригинальному.

\paragraph{Объявление типового параметра}
Для \((A : S) \rightarrow T\) рассмотрим 2 способа:
\begin{enumerate}
\item \( \forall a.\ () \rightarrow T \)
\item \( \forall (a :: K).\ T,\quad K\) --- \(S\), где \(Set_0\) заменяется на \(*\)
\end{enumerate}

Первый используется в Agda 2.3.4, как описано в \ref{text:agda-compiled-export}.
Второй --- в данной работе.

Добавление аргумента \(()\) делает терм более похожим на Agda, так как в ней
требуется передача типового аргумента терму и этот способ используется в
MAlonzo при генерации кода. С другой стороны, второй способ является более
естественным при использовании из Haskell, но требует написания оберток над
кодом, сгенерированным MAlonzo, которые будут выполнять преобразование между
2-мя типами.

\paragraph{Типовой атом}
Разбивается на 4 случая:
\begin{enumerate}
\item \textbf{Типовой параметр}

Заменяется на переменную из соответствующего объявления
\item \textbf{<<Экспортированный>> тип}

Заменяется на тип, в который экспортировали
\item \textbf{<<Импортированный>> тип}

Заменяется на тип, из которого импортировали с помощью \texttt{COMPILED\_DATA} или
\texttt{COMPILED\_TYPE}
\item \textbf{Встроенный тип}

Встроенные типы, получаемые из постулатов: \texttt{INTEGER}, \texttt{FLOAT},
\texttt{CHAR}, \texttt{STRING}, \texttt{IO} - заменяются соответственно на
\(Int\), \(Float\), \(Char\), \(String\), \(IO\), так как MAlonzo использует
ровно эти типы вместо постулатов.

Встроенные типы вроде \texttt{LIST} в MAlonzo просто получают функции
по преобразованию между \([]\) и типом, сгенерированным по обычным правилам
и таким образом эквивалентны генерированию экспорту типов вместе с конструкторами.
\end{enumerate}

\subsubsection{Обертки для функций}\label{sec:implementation-wrappers}

Необходимо по типу Agda и по терму из MAlonzo сгенерировать терм Haskell с типом
из \ref{sec:implementation-functions}, использующий терм из MAlonzo.

Рассмотрим преобразование \(Wrap^n\llbracket AgdaType \rrbracket(Term) = Term\).
\(n\) задает уровень вложенности. Тогда генерируемая
обертка для функции с типом из Agda \(AgdaType\) и термом из MAlonzo \(MAlonzoTerm\)
будет выражаться \(Wrap^0\llbracket AgdaType \rrbracket(MAlonzoTerm)\).

\begin{description}
\item[\(Wrap^n\llbracket (x : T_1) \rightarrow T_2 \rrbracket(t) =
   \lambda x.\ Wrap^n\llbracket T_2 \rrbracket(t\ (
   Wrap^{n+1}\llbracket T_1 \rrbracket(x)))\):]
\(x\) имеет уровень вложенности на 1 больше по определению.
\item[\(Wrap^{2k}\llbracket (A : S) \rightarrow T \rrbracket(t) =
   Wrap^{2k}\llbracket T \rrbracket(t\ ())\):]
Если уровень вложенности четный, то \(t\) --- терм из MAlonzo и необходимо
вместо параметра типа подставить \(()\)

\item[\(Wrap^{2k + 1}\llbracket (A : S) \rightarrow T \rrbracket(t) =
   Wrap^{2k + 1}\llbracket T \rrbracket(\lambda \_.\ t)\):]
Если уровень вложенности нeчетный, то \(t\) --- терм из внешнего кода и необходимо
пропустить параметр типа, который MAlonzo попытается туда подставить.

\item[\(Wrap^n\llbracket X\ args\dots \rrbracket(t) = \texttt{unsafeCoerce}\ t\):]
На текущий момент \(X\) либо типовой параметр, либо экспортированный тип(то
есть представляется, как \textbf{newtype}-обертка над типом из MAlonzo),
либо импортированный тип или встроенный тип(то есть MAlonzo использует его
вместо генерации нового). Все случаи позволяют использовать \texttt{unsafeCoerce}.
\end{description}

Из последнего пункта вытекает проблема с использованием встроенных типов
\texttt{LIST} и \texttt{BOOL}: MAlonzo генерирует свои типы для них, а также
биекции между ними и \([]\), \(Bool\) соответственно. Таким образом, использовать
\(unsafeCoerce\) нельзя: конкретный пример в \ref{sec:implementation-data-trouble}.

\subsubsection{Проблема с генерированием типов данных}
\label{sec:implementation-data-trouble}

Рассмотрим код на Agda:

\begin{align*}
&\mathbf{open\ import}\ Data.Maybe\\
\\
&\mathbf{data}\ List\ \{l\}\ (A : Set\ l) : Set\ l\ \mathbf{where}\\
&\quad nil : List A\\
&\quad cons : A \rightarrow List A \rightarrow List A\\
\\
&empty : \forall \{l\}\ \{A : Set\ l\} \rightarrow List\ A\\
&empty = nil\\
\\
&head : \forall \{l\}\ \{A : Set\ l\} \rightarrow List\ A \rightarrow Maybe\ A\\
&head\ nil = nothing\\
&head\ (cons\ x\ \_) = just\ x\\
\\
&\mathbf{data}\ Useless\ (A : Set) : Set_1\ \mathbf{where}\\
&\quad useless : \{B : Set\} \rightarrow (B \rightarrow A) \rightarrow B
   \rightarrow Useless\ A\\
\\
&\mathbf{data}\ Either (A\ B : Set) : Set\ \mathbf{where}\\
&\quad left : A \rightarrow Either\ A\ B\\
&\quad right : B \rightarrow Either\ A\ B\\
\\
&add1 : \forall \{A\} \rightarrow Useless\ A \rightarrow List\ (Useless\ A)
   \rightarrow List\ (Useless\ A)\\
&add1 = cons\\
\\
&add2 : \forall \{A B\} \rightarrow Either\ A\ B \rightarrow List\ (Either\ A\ B)
   \rightarrow List\ (Either\ A\ B)\\
&add2 = cons
\end{align*}

По нему получится примерно следующий код на Haskell, сгенерируемый MAlonzo:

\begin{align*}
&\mathbf{data}\ T1\ a0\ a1 = C2\ |\ C3\ a0\ a1\\
\\
&d4 = \texttt{unsafeCoerce}\ (\lambda v0\ v1 \rightarrow C2)\\
\\
&d5\ v0\ v1\ C2 = \texttt{unsafeCoerce}\ Nothing\\
&d5\ v0\ v1\ (C3\ v2\ v3) = \texttt{unsafeCoerce}\ (Just\ (\texttt{unsafeCoerce}\ v2))\\
\\
&\mathbf{data}\ T6\ a0\ a1\ a2 = C7\ a0\ a1\ a2\\
\\
&\mathbf{data}\ T8\ a0 = C9\ a0\ |\ C10\ a0\\
\\
&d11 = \texttt{unsafeCoerce} (\lambda v0 \rightarrow C3)\\
\\
&d12 = \texttt{unsafeCoerce} (\lambda v0\ v1 \rightarrow C3)\\
\end{align*}

Если потребовать экспортировать все написанные имена\footnote{
Нужно представить, что есть поддержка \(Set\ l\)(нужна только для \(Useless\))} и
потребовать, чтобы \(Useless\) и \(Either\) были экспортированы полностью, то
можно получить примерно следующий код:

\begin{align*}
&\mathbf{newtype}\ List\ a = List\ (\mathbf{forall}\ b0\ b1.\ T1\ b0\ b1)\\
\\
&empty :: List\ a\\
&empty = \texttt{unsafeCoerce}\ (d4\ ()\ ())\\
\\
&head' :: List\ a \rightarrow Maybe\ a\\
&head' = \lambda xs \rightarrow \texttt{unsafeCoerce}\ (d13\ ()\ ()\ (
   \texttt{unsafeCoerce}\ xs))\\
\\
&\mathbf{data}\ Useless\ a\ \mathbf{where}\\
&\quad Useless :: (b \rightarrow a) \rightarrow b \rightarrow Useless\ a\\
&tUselessToMA :: Useless\ a \rightarrow T6\ a0\ a1\ a2\\
&tUselessToMA\ (Useless\ a0\ a1) = \texttt{unsafeCoerce}\ (C7\ ()\ (
   \lambda x \rightarrow\\
&\quad\texttt{unsafeCoerce}\ (\texttt{unsafeCoerce}\ a0\ (
   \texttt{unsafeCoerce}\ x)))\ (\texttt{unsafeCoerce}\ a1))\\
&tUselessFromMA :: T6\ a0\ a1\ a2 \rightarrow Useless\ a\\
&tUselessFromMA\ (C7\ a0\ a1\ a2) = \texttt{unsafeCoerce}\ (Useless\ (
   \lambda x \rightarrow\\
&\quad\texttt{unsafeCoerce}\ (\texttt{unsafeCoerce}\ a1\ (
   \texttt{unsafeCoerce}\ x)))\ (\texttt{unsafeCoerce}\ a2))\\
\\
&\mathbf{data}\ Either'\ a\ b\ \mathbf{where}\\
&\quad Right' :: b \rightarrow Either'\ a\ b\\
&\quad Left' :: a \rightarrow Either'\ a\ b\\
&tEitherToMA :: Either\ a\ b \rightarrow T8\ a0\\
&tEitherToMA\ (Left'\ a0) = \texttt{unsafeCoerce}\ (C9\ (\texttt{unsafeCoerce}\ a0))\\
&tEitherToMA\ (Right'\ a0) = \texttt{unsafeCoerce}\ (C10\ (\texttt{unsafeCoerce}\ a0))\\
&tEitherFromMA :: T8\ a0 \rightarrow Either\ a\ b\\
&tEitherFromMA\ (C9\ a0) = \texttt{unsafeCoerce}\ (Left'\ (\texttt{unsafeCoerce}\ a0))\\
&tEitherFromMA\ (C10\ a0) = \texttt{unsafeCoerce}\ (Right'\ (\texttt{unsafeCoerce}\ a0))\\
\\
&add1 :: Useless\ a \rightarrow List\ (Useless\ a) \rightarrow List\ (Useless\ a)\\
&add1 = \lambda x\ xs \rightarrow \texttt{unsafeCoerce} (d11\ () \ (
   tUselessToMA\ x)\ (\texttt{unsafeCoerce}\ xs))\\
\\
&add2 :: Either'\ a\ b \rightarrow List\ (Either'\ a\ b) \rightarrow
   List\ (Either'\ a\ b)\\
&add2 = \lambda x\ xs \rightarrow \texttt{unsafeCoerce} (d12\ ()\ () \ (
   tEitherToMA\ x)\ (\texttt{unsafeCoerce}\ xs))\\
\end{align*}

А теперь 2 функции из внешнего кода:

\begin{align*}
&test1 :: Maybe\ (Useless\ Int)\\
&test1 = head'\ (add1\ (Useless\ read\ \texttt{``3''})\ empty)\\
\\
&test2 :: Maybe\ (Either'\ Int\ Char)\\
&test2 = head'\ (add2\ (Left'\ 3)\ empty)
\end{align*}

В обоих случаях \(add1\) и \(add2\) используют соответственно \(tUselessToMA\) и
\(tEitherToMA\) для конвертирования в тип MAlonzo, но \(head'\) использует
\texttt{unsafeCoerce} вместо \(tUselessFromMA\) и \(tEitherFromMA\) для
конвертирования обратно. Можно заметить, что \(Either'\) задан с конструкторами
в другом порядке и поэтому тип \(Either'\) гарантировано отличается от \(T8\)
и результат \(test2\): \(Just\ (Right'\ `\backslash ETX')\). В случае с
\(Useless\) проблема: он лежит в \(Set_1\), а трансформация поддерживает только
\(Set_0\). Поэтому код интерфейса был написан вручную, как если бы была возможность
его сгенерировать. Здесь \texttt{unsafeCoerce} не применим, так как в конструкторе
есть параметр типа в качестве аргумента, которого не будет в интерфейсе.

Таким образом, необходимо поддерживать следующий инвариант: тип данных в интерфейсе
должен иметь внутреннее представление идентичное типу, генерируемому MAlonzo. Значит,
необходимо либо делать \texttt{newtype}-обертки, либо менять тип, генерируемый
MAlonzo.

\paragraph{Неприменимость проблемы для функций}

Выше утверждалось, что <<тип данных в интерфейсе должен иметь внутренее представление,
идентичное типу, генерируемому MAlonzo>>. Тип функций тоже в некотором роде тип данных
(например, тип пары: \((a \rightarrow b \rightarrow c) \rightarrow c\)) и если функция
полиморфна, то этот тип будет иметь разные представления в интерфейсе и внутри MAlonzo.
Рассматривая пример выше, напишем еще одну функцию:

\begin{align*}
&\text{Agda:}\\
&add3 : (\forall \{A\} \rightarrow A \rightarrow A) \rightarrow List\ (\forall \{A\}
   \rightarrow A \rightarrow A) \rightarrow List\ (\forall \{A\} \rightarrow A
   \rightarrow A)\\
&add3 = cons\\
\\
&\text{MAlonzo:}\\
&d13 = \texttt{unsafeCoerce}\ C3\\
\\
&\text{Интерфейс:}\\
&add3 :: (\mathbf{forall}\ a.\ a \rightarrow a) \rightarrow
   List\ (\mathbf{forall}\ a.\ a \rightarrow a) \rightarrow
   List\ (\mathbf{forall}\ a.\ a \rightarrow a)\\
&add3 = \lambda f\ fs \rightarrow \texttt{unsafeCoerce}\ (d13\\
&\quad(\lambda \_\ x \rightarrow \texttt{unsafeCoerce}\ (
      f\ (\texttt{unsafeCoerce}\ x)))
   \ (\texttt{unsafeCoerce}\ fs))\\
\\
&\text{Тест:}\\
&test3 :: Int\\
&test3 =\\
&\quad\mathbf{case}\ head'\ (add3\ id\ empty)\ \mathbf{of}\\
&\quad\ Nothing \rightarrow \texttt{error}\ \text{``Impossible''}\\
&\quad\ Just\ f \rightarrow f\ 3
\end{align*}

Ошибка должна быть та же самая: один раз функцию упаковали в список, разложив ее
на аргументы, достали из него с помощью \texttt{unsafeCoerce}. Но такой код
нескомпилируется: он требует расширения языка \texttt{ImpredicativeTypes},
без которого у типа данных типовые параметры обязаны быть мономорфными\cite{SPJ11}. А поскольку
только полиморфные функции отличаются по структуре между интерфейсом и MAlonzo,
то проблема решается автоматически.

Что если теперь вместо специально построенного \(List\) использовать чистое
\(\lambda\)-исчисление для эмуляции типов данных. Сразу рассмотрим типы функций
в интерфейсе:

\begin{align*}
&lnil :: a \rightarrow b\\
&lcons :: a \rightarrow ((a \rightarrow b \rightarrow c) \rightarrow d \rightarrow b)
   \rightarrow (a \rightarrow b \rightarrow c) \rightarrow d \rightarrow c\\
&lhead :: ((c \rightarrow b \rightarrow c) \rightarrow (d \rightarrow e \rightarrow e)
   \rightarrow a) \rightarrow a\\
&ladd :: (\mathbf{forall}\ a.\ a \rightarrow a) \rightarrow
   (((\mathbf{forall}\ a.\ a \rightarrow a) \rightarrow b \rightarrow c)
   \rightarrow d \rightarrow b) \rightarrow \\
&\quad((\mathbf{forall}\ a.\ a \rightarrow a) \rightarrow b \rightarrow c)
   \rightarrow d \rightarrow c\\
\\
&test = lhead\ (ladd\ id\ lnil)
\end{align*}

Получится та же самая ошибка компиляции: требуется импредикативность. На самом деле,
ошибка с импредикативностью при использовании \(List\) появляется из-за аналогичных причин.
