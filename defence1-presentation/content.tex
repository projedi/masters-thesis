\frame {
   \frametitle{Мотивирующий пример}
   \begin{align*}
      &\texttt{elemAt :: [a] \(\rightarrow\) Int \(\rightarrow\) a}\\
      &\texttt{elemAt [] 3 = {\color{red}???}}\\
      \\
      &\texttt{elemAt :: (xs :: [a]) \(\rightarrow\) (n :: Int) \(\rightarrow\)}\\
      &\texttt{\ \ \ (n < (length xs)) \(\rightarrow\) a}\\
      &\texttt{\color{red}elemAt [] 3 (???) = \dots}
   \end{align*}
}

\frame {
   \frametitle{Зависимые типы}
   System F:
   \begin{align*}
      \mathbf{Term} &::= \mathbf{Var}\ |\ \lambda x.\ \mathbf{Term}(x)\ |
                         \ \mathbf{Term}\ \mathbf{Term}\ |
                         \ (\mathbf{Term},\ \mathbf{Term}) \\
      \mathbf{Type} &::= \mathbf{TVar}\ |\ \mathbf{Type} \rightarrow \mathbf{Type}\ |
                         \ \forall x.\ \mathbf{Type}(x)\ |
                         \ \mathbf{Type} \times \mathbf{Type}
   \end{align*}
   \[
      \mathbf{\Gamma} \vdash \mathbf{Term} : \mathbf{Type},\quad \mathbf{\Gamma} = \mathbf{Var} : \mathbf{Type},\dots
   \]
   \pause
   Зависимые типы:
   \begin{align*}
      \mathbf{Term} ::=&\ \mathbf{Var}\\
                      |&\ \mathbf{Term}\ \mathbf{Term}\\
                      |&\ \lambda x.\ \mathbf{Term}(x)\ |\ (x : \mathbf{Term}) \rightarrow \mathbf{Term}(x)\\
                      |&\ (\mathbf{Term}, \mathbf{Term})\ |\ (x : \mathbf{Term}) \times \mathbf{Term}(x)\\
                      |&\ \mathbf{Set}
   \end{align*}
   \[
      \mathbf{\Gamma} \vdash \mathbf{Term} : \mathbf{Term},\quad \mathbf{\Gamma} = \mathbf{Var} : \mathbf{Term},\dots
   \]
}

\frame {
   \frametitle{Agda}
   Язык с зависимыми типами и синтаксисом, похожим на Haskell.
}

\frame {
   \frametitle{Agda - примеры}
   \begin{align*}
      &{\color{red}\mathbf{data}} \textit{ работает аналогично GADT в Haskell}\\
      &\mathbf{data}\texttt{ List (A : Set) : Set }\mathbf{where}\\
      &\texttt{\ \ \ <> : List A}\\
      &\texttt{\ \ \ \_::\_ : A \(\rightarrow\) List A \(\rightarrow\) List A}\\
      \\
      &\texttt{\color{red}\{\dots\}}
         \textit{ --- необязательный аргумент(компилятор сам подставит)} \\
      &\texttt{list-length : \{A : Set\} \(\rightarrow\) List\ A \(\rightarrow\) Nat}\\
      &\texttt{list-length <> = 0}\\
      &\texttt{list-length (x :: xs) = list-length xs + 1}\\
   \end{align*}
}
\frame {
   \frametitle{Agda - примеры}
   \begin{align*}
      &\mathbf{data}\texttt{ Vec (A : Set) : Nat \(\rightarrow\) Set }\mathbf{where}\\
      &\texttt{\ \ \ nil : Vec\ A\ 0}\\
      &\texttt{\ \ \ cons : \{n : Nat\} \(\rightarrow\) A \(\rightarrow\) Vec\ A\ n
         \(\rightarrow\) Vec A (n + 1)}\\
      \\
      &\texttt{list-to-vec : \{A : Set\} \(\rightarrow\) (xs : List A) \(\rightarrow\)}\\
      &\texttt{\ \ \ Vec A ({\color{red}list-length xs})}\\
      &\texttt{list-to-vec\ <>\ = \text{nil}}\\
      &\texttt{list-to-vec\ (x :: xs) = \text{cons}\ x\ (\text{list-to-vec}\ xs)}\\
   \end{align*}
}
\frame {
   \frametitle{Agda - примеры}
   \begin{align*}
      &\texttt{\color{red}\{a1 a2 : A\}(b : B) \(\rightarrow\) C}
         \textit{ --- синтаксический сахар для }\\
      &\texttt{\ \ \ \{a1 : A\} \(\rightarrow\) \{a2 : A\} \(\rightarrow\)
         (b : B) \(\rightarrow\) C}\\
      \\
      &\texttt{zip-vec : \{A B : Set\} \{n : Nat\} \(\rightarrow\) Vec A n
         \(\rightarrow\) Vec B n \(\rightarrow\)}\\
      &\texttt{\ \ \ Vec (A \(\times\) B) n}\\
      &\texttt{zip-vec nil nil = nil}\\
      &\texttt{zip-vec (cons x xs) (cons y ys) =}\\
      &\texttt{\ \ \ cons (x , y) (zip-vec xs ys)}\\
      &\color{red}\texttt{zip-vec nil (cons y ys) = \dots}\\
      &\color{red}\texttt{zip-vec (cons x xs) nil = \dots}\\
      &\textit{эти 2 клоза даже написать нельзя - типы не сойдутся}
   \end{align*}
}

\frame {
   \frametitle{Agda}
   \begin{align*}
      &\mathbf{data}\ \texttt{\_<\_ : Nat \(\rightarrow\) Nat \(\rightarrow\) Set}
         \ \mathbf{where}\\
      &\ \ \ \texttt{leq-zero : \{n : Nat\} \(\rightarrow\) (0 < n + 1)}\\
      &\ \ \ \texttt{leq-suc : \{n m : Nat\} \(\rightarrow\) (n < m) \(\rightarrow\)
         (n + 1 < m + 1)}\\
      \\
      &\texttt{elemAt : \{A : Set\}(xs : List\ A)(n : Nat) }\rightarrow\\
      &\texttt{\ \ \ (n < list-length xs) \(\rightarrow\) A}\\
      &\texttt{elemAt <> \_ ()}\\
      &\texttt{elemAt (x :: xs) 0 \_ = x}\\
      &\texttt{elemAt (\_ :: xs) (n + 1) (leq-suc p) =
         elemAt xs n p}\\
   \end{align*}
}

\frame {
   \frametitle{Компилятор MAlonzo}
   \begin{align*}
      &\mathbf{data}\texttt{ Vec (A : Set) : Nat \(\rightarrow\) Set }\mathbf{where}\\
      &\texttt{\ \ \ nil : Vec A 0}\\
      &\texttt{\ \ \ cons : \{n : Nat\} \(\rightarrow\) A \(\rightarrow\) Vec A n
         \(\rightarrow\) Vec A (n + 1)}\\
      &\texttt{vec-map : \{A B : Set\}\{n : Nat\} \(\rightarrow\) (A \(\rightarrow\) B)
         \(\rightarrow\)}\\
      &\texttt{\ \ \ Vec A n \(\rightarrow\) Vec B n}\\
      &\texttt{vec-map f nil = nil}\\
      &\texttt{vec-map f (cons x xs) = cons (f x) (vec-map f xs)}\\
      \\
      &\mathbf{data}\texttt{ T1 a0 a1 a2 = C2 | C3 a0 a1 a2}\\
      &\texttt{d4 v0 v1 v2 v3 (C2) = cast C2}\\
      &\texttt{d4 v0 v1 v2 v3 (C3 v4 v5 v6) = cast
         (C3\ (cast v4) \dots)}\\
      &\mathbf{ghci}\texttt{>:t d4}\\
      &\texttt{d4 :: a \(\rightarrow\) a1 \(\rightarrow\) a2
         \(\rightarrow\) a3 \(\rightarrow\) (T1 t t1 t2) \(\rightarrow\) b}
   \end{align*}
}

\frame {
   \frametitle{Coq}
   \begin{align*}
      &\mathbf{data}\texttt{ Vec a = Nil | Cons Nat a (Vec a)}\\
      \\
      &\texttt{vecMap :: Nat \(\rightarrow\) (a1 \(\rightarrow\) a2) \(\rightarrow\)
         Vec a1 \(\rightarrow\) Vec a2}\\
      &\texttt{vecMap n f v =}\\
      &\ \ \ \mathbf{case}\ \texttt{v}\ \mathbf{of}\\
      &\texttt{\ \ \ \ Nil \(\rightarrow\) Nil}\\
      &\texttt{\ \ \ \ Cons n0 x xs \(\rightarrow\) Cons n0 (f x)
         (vecMap n0 f xs)}\\
   \end{align*}
}

\frame {
   \frametitle{Что хочется}
   \begin{block}{Идея}
   Нужно сохранить инварианты, поддерживаемые системой типов.
   \vspace{1em}
   \begin{enumerate}
   \item Экспортируем весь код в Haskell с помощью MAlonzo.
   \item
      Для выбранных имен пытаемся найти соответствующие типы на Haskell
      и генерируем обертки над скомпилированным кодом.
   \end{enumerate}
   \end{block}
}

\frame {
   \frametitle{Простое решение}
   \begin{Lem}
   Типы Agda, лежащие в System F, представимы в Haskell и имеют ту же семантику.
   \end{Lem}
   {\color{red} Формально не доказывал, но, вроде, очевидно.}\\
   \vspace{1.5em}
   Указание на <<экспорт>> дается с помощью прагм\\
   \texttt{\{-\# EXPORT agda-name haskell-name \#-\}}

   \begin{itemize}
   \item \(\mathbf{data}\texttt{ T (A : Set) : Set \(\rightarrow\) Set }\mathbf{where}\)
      породит
      \(\mathbf{newtype}\texttt{ T a0 a1 = <hidden>}\)
   \item \texttt{f : \{A : Set\} \(\rightarrow\) (\{B : Set\} \(\rightarrow\) B)
      \(\rightarrow\) A} породит\\
      \texttt{f :: \(\forall\)a0. (\(\forall\)b0. b0) \(\rightarrow\) a0}
   \end{itemize}
}

\frame {
   \frametitle{Недоработки}
   Некоторые типы можно экспортировать вместе с конструкторами:\\
   \texttt{\{-\# EXPORT\_DATA agda-name hs-name hs-con-name \dots\ \#-\}}\\
   Проблема оборачивания функций:
   \begin{itemize}
   \item \textbf{newtype} гарантирует, что внутреннее представление совпадает с
      оборачиваемым типом \(\Rightarrow\) \texttt{unsafeCoerce}.
   \item \textbf{data} не может дать никаких похожих гарантий \(\Rightarrow\)
      генерирование разбора по конструкторам.
   \end{itemize}

   Из этого же --- невозможно экспортировать \texttt{f}:
   \begin{align*}
   &\mathbf{newtype}\texttt{ T1 a0 = \dots}\\
   &\mathbf{data}\texttt{ T2 a0 = \dots}\\
   \\
   &\texttt{f : \{A : Set\} \(\rightarrow\) T1 (T2 A) \(\rightarrow\) A}\\
   &\color{red}\texttt{f :: \(\forall\)a. T1 (T2 a) \(\rightarrow\) a}
   \end{align*}
}

\frame {
   \frametitle{Недоработки}
   \begin{itemize}
      \item Haskell поддерживает \(\mathbf{data}\texttt{ T (a :: * \(\rightarrow\) *)}\)
         аналог в Agda \(\mathbf{data}\texttt{ T (A : Set \(\rightarrow\) Set)}\)
      \item У Agda есть FFI в Haskell и есть набор встроенных типов. При компиляции
         они обрабатываются специально.
      \item В Agda eсть параметризованные модули:
         \(\mathbf{module}\texttt{ Main (A : Set) }\mathbf{where}\)
      \item Сейчас все экспорты генерируются в том же модуле, что и скомпилированный
         код(\texttt{MAlonzo.Code.Module.Name}. Лучше все это вынести в отдельное
         поддерево \texttt{MAlonzo.Export.Module.Name}, чтобы экспортировать
         \textbf{newtype} абстрактно(скрыв реализацию от пользователя) и чтобы
         автоматически можно было сгенерировать документацию без мусора.
   \end{itemize}
}

\frame {
   \frametitle{Что еще можно попробовать}
   На Haskell можно проэмулировать некоторые зависимые типы
   \begin{enumerate}
      \item Если \texttt{n} не используется внутри \texttt{f}, то
         \begin{align*}
         &\texttt{f : \{A B : Set\}\{n : Nat\} \(\rightarrow\) Vec A n \(\rightarrow\)}\\
         &\texttt{\ \ \ Vec B n \(\rightarrow\) Vec (A \(\times\) B) n}
         \end{align*}
         эмулируется без проблем.
      \item есть трюк, позволяющий эмулировать типы вроде
         \[\texttt{f : \{A : Set\}(n : Nat) \(\rightarrow\) A \(\rightarrow\) Vec A n}\]
      \item \dots
   \end{enumerate}
}

\frame {
   \frametitle{Сложности}
   \begin{itemize}
      \item В Agda все функции должны быть полными и завершающимися.
         \begin{align*}
            &\texttt{f : \{A\ B : Set\}(g : A \(\rightarrow\) B)(proof : Bijection g)
               \(\rightarrow\)}\\
            &\texttt{\ \ \ B \(\rightarrow\) A}
         \end{align*}
         Что делать, если мы передадим \texttt{g}, которая зацикливается?
      \item На самом деле в Agda могут быть не завершающиеся функции с
         использованием коиндукции.
   \end{itemize}
}

\frame {
   \center{\Huge Q\&A}
}
