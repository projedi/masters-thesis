\frame {
   \frametitle{Формальная верификация}

   \begin{itemize}
   \item Способ доказать, что программа решает поставленную задачу.
         Тестирование только может показать это на ограниченном наборе данных.
   \item Agda --- функциональный язык программирования с системой типой, позволяющей
         строить и проверять логические формулы над элементами языка.
   \end{itemize}

   TODO: Super small example
}

\frame {
   \frametitle{Использование верифицированного кода}

   \begin{enumerate}
   \item Использовать Agda для написания приложений целиком.
         \begin{description}
         \item[\(+\)] Можно верифицировать больше кода.
         \item[\(-\)] Не Тьюринг-полный язык.
         \end{description}
   \item По коду на Agda генерировать код на другом языке (<<Экстракция программ>>).
         \begin{description}
         \item[\(+\)] Удобнее писать <<реальный>> код.
         \item[\(-\)] Необходимо поддерживать корректность кода при трансляции.
         \end{description}
   \end{enumerate}
}

\frame {
   \frametitle{Экстракция программ}

   \begin{itemize}
   \item Уже есть транслятор: MAlonzo компилирует Agda через трансляцию в Haskell.
   \item Необходимо уметь запрещать прямой доступ к функциям со сложными предусловиями
         на аргументы.
   \end{itemize}
}

\frame {
   \frametitle{Существующие решения}

   \begin{description}
   \item[Coq 8.4pl3]
      Экстракция программ\footnote{P. Letouzey. \emph{A New Extraction for Coq}. 2002}.
      Из кода стираются все доказательства и инварианты.
   \item[Agda 2.3.2.2]
      Компилятор MAlonzo\footnote{M. Benke. \emph{Alonzo --- a compiler for Agda}. 2007}.
      Написан для генерации исполняемых файлов. Генерирует имена вида буква+число,
      теряет почти всю информацию о типах.
   \item[Agda 2.4.0]
      Появилась возможность давать пользовательские имена функциям
      и генерировать для них разумные типы.
   \end{description}
}

\frame {
   \frametitle{Цели и задачи}

   \begin{block}{Цель работы}
      Разработать механизм для MAlonzo, генерирующий интерфейс на Haskell к коду
      на Agda, использование которого не позволит нарушить инварианты, поддерживаемые
      Agda.
   \end{block}

   Задачи:
   \begin{enumerate}
   \item Провести анализ методов трансляции и принципов генерации кода в компиляторе
         MAlonzo.
   \item Разработать механизм генерации интерфейса на Haskell к сгенерированному
         MAlonzo коду, не позволяющий нарушить инварианты, поддерживаемые Agda.
   \item Доказать корректность генерируемого интерфейса.
   \item Реализовать поддержку механизма экстракции в компиляторе MAlonzo и
         провести тестирование этой реализации.
   \end{enumerate}
}

\frame {
   \frametitle{Обзор реализации}

   \begin{itemize}
   \item Генерируемый интерфейс --- обертки над кодом, сгенерированным MAlonzo,
         которые имеют согласованные с Agda типы.
   \item Введена прагма \texttt{\{-\# EXPORT \(AgdaName\) \(HaskellName\) \#-\}}.
         Если сущность \(AgdaName\) представима в Haskell и \(HaskellName\) ---
         разрешенное имя, то генерируется соответствующая обертка.
   \item Модуль с генерируемым интерфейсом помещается в поддерево
         \(MAlonzo.Export\), отдельно от MAlonzo-генерируемого кода
         в \(MAlonzo.Code\).
   \end{itemize}
}

\frame {
   \frametitle{Подробности реализации}
   \framesubtitle{Тип данных и все его конструкторы выразимы}

   Agda:
   \begin{align*}
   &\mathbf{data}\ List\ (A : Set) : Set\ \mathbf{where}\\
   &\quad nil : List\ A\\
   &\quad cons : A \rightarrow List\ A \rightarrow List\ A
   \end{align*}

   Haskell:
   \[
   \mathbf{data}\ List\ (a :: *) = Nil\ |\ Cons\ a\ (List\ a)
   \]

   \begin{itemize}
   \item При реализации требуется, чтобы терм MAlonzo имел идентичную внутреннюю структуру
         терму интерфейса.
   \item Необходимо подменять тип, генерируемый MAlonzo.
   \end{itemize}
}

\frame {
   \frametitle{Подробности реализации}
   \framesubtitle{Тип данных выразим, но хотя бы один конструктор не выразим}

   Agda:
   \begin{align*}
   &\mathbf{data}\ IsEqual\ (A : Set) : Set\ \mathbf{where}\\
   &\quad yes : (x\ y : A) \rightarrow x \equiv y \rightarrow IsEqual\ A\\
   &\quad no : (x\ y : A) \rightarrow x \not\equiv y \rightarrow IsEqual\ A
   \end{align*}

   Haskell:
   \begin{align*}
   &\mathbf{newtype}\ IsEqual\ (a :: *) =\\
   &\quad IsEqual\ (\forall b_0 \cdots b_3.\ MAlonzoType\ b_0 \cdots b_3)
   \end{align*}
   \begin{itemize}
   \item Позволяет использовать \texttt{unsafeCoerce} для трансформации между
         этим типом и типом из MAlonzo.
   \item Такой же подход используется сейчас и для простых типов.
   \end{itemize}
}

\frame {
   \frametitle{Подробности реализации}
   \framesubtitle{Функции}

   Agda:
   \begin{align*}
   &identity : \{A : Set\} \rightarrow A \rightarrow A\\
   &identity\ x = x
   \end{align*}

   Haskell:
   \begin{align*}
   &identity :: \mathbf{forall}\ (a :: *).\ a \rightarrow a\\
   &identity = \lambda x \rightarrow \texttt{unsafeCoerce}\\
   &\quad(mAlonzoTerm\ Unit\ (\texttt{unsafeCoerce}\ x))
   \end{align*}

   \begin{itemize}
   \item MAlonzo оставляет типовой параметр аргументом функции.
   \item Из генерируемого интерфейса они убираются.
   \end{itemize}
}

\frame {
   \frametitle{Доказательство корректности}

   Методом структурной индукции выполняемые трансформации были показаны корректными.
}

\frame {\label{meta:slidecount}
   \frametitle{Выводы}

   Таким образом:
   \begin{enumerate}
   \item Экстракция кода из Agda в Haskell, сохраняющая семантику, возможна для широкого класса типов данных и функций Agda.
   \item Разработан способ генерировать безопасный интерфейс на Haskell к коду на Agda.
   \item Доказана его корректность.
   \end{enumerate}
}

\frame {
   \frametitle{Agda 2.4.0}

   \begin{itemize}
   \item Прагма \texttt{\{-\# COMPILED\_EXPORT \(AgdaName\) \(HaskellName\) \#-\}}.
   \item Подменяется имя функций (переименовывание типов данных не поддерживается), генерируемых MAlonzo.
   \item Функции требуют \(Unit\) на место параметров типов.
   \end{itemize}
}

\frame {
   \frametitle{Что дальше}

   \begin{itemize}
   \item Необходимо сделать экспортирования типов данных целиком: подменять код MAlonzo для
         тиопв данных и конструкторов.
   \item В Agda есть способ симулировать классы типов из Haskell с помощью \textbf{record} --- можно
         попробовать генерировать соответствующие классы типов и их реализации.
   \item Существует множество расширений системы типов Haskell, которые позволяют в той или иной
         степени реализовывать зависимые типы --- их использование позволит выражать больше типов Agda
         в Haskell.
   \end{itemize}
}

\frame {
   \frametitle{\textbf{data} вместе с биекциями}

   \begin{align*}
   &\mathbf{newtype}\ List\ a,\quad\mathbf{data}\ T = In\ Int\ |\ Ch\ Char\\
   &trTtoT1 :: T \rightarrow T1,\quad trT1toT :: T1 \rightarrow T,\quad empty :: List\ a\\
   \\
   &add :: T \rightarrow List\ T \rightarrow List\ T\\
   &add = \lambda x\ xs.\ \texttt{unsafeCoerce}\ (d7\ (trTtoT1\ x)
      \ (\texttt{unsafeCoerce}\ xs))\\
   \\
   &head :: List\ a \rightarrow a\\
   &head = \lambda xs.\ {\color{red}\texttt{unsafeCoerce}}\ (d8
      \ (\texttt{unsafeCoerce}\ xs))\\
   \\
   &test = head\ (add\ (In\ 3)\ empty)
   \end{align*}

  Для корректной работы \(head\) обязан был вызвать \(trT1toT\) вместо
  \texttt{unsafeCoerce}.
}
