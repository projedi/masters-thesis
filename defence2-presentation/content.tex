\frame {
   \frametitle{Формальная верификация}

   \begin{itemize}
   \item Необходимо уметь убеждаться, что написанная программа решает
         поставленную задачу.
   \item Тестирование не может показать, что программа верна для всех случаев
         (если, конечно, нельзя сделать полный перебор).
   \item Формальная верификация позволяет сравнить программу с формальной
         математической моделью и доказать их эквивалентность на всех входных данных.
   \end{itemize}
}

\frame {
   \frametitle{Agda}

   \begin{itemize}
   \item Один из способов формально верифицировать --- строить формулы достаточно
         мощной логики над элементами программы и проверять их на этапе компиляции.
   \item Agda --- функциональный язык программирования, система типов которого позволяет строить
         формулы на языке предикативной конструктивной логики.
   \end{itemize}
}

\frame {
   \frametitle{Использование верифицированного кода}

   Написание верифицированного алгоритма недостаточно --- необходимо еще
   использовать этот код из <<реальных>> приложений. Подходы:
   \begin{enumerate}
   \item Использовать Agda для написания приложений целиком.
         \begin{description}
         \item[\(+\)] Можно верифицировать больше кода.
         \item[\(-\)] Не Тьюринг-полный язык.
         \end{description}
   \item По коду на Agda генерировать код на другом языке
         \begin{description}
         \item[\(+\)] Удобнее писать <<реальный>> код.
         \item[\(-\)] Необходимо поддерживать корректность кода при трансляции.
         \end{description}
   \end{enumerate}

   Второй пункт называется <<экстракция программ>> и используется в системе Coq.
}

\frame {
   \frametitle{Постановка задачи}

   \begin{block}{Задача}
      По коду на Agda получить код на Haskell, который можно использовать
      из программы на Haskell, не нарушая внутренние инварианты, поддерживаемые Agda.
   \end{block}

   Целевым языком выбран Haskell по нескольким причинам:
   \begin{itemize}
   \item Уже есть транслятор: MAlonzo компилирует Agda через трансляцию в Haskell
   \item Языки синтаксически похожи
   \item Типы в Haskell --- подмножество типов в Agda
   \end{itemize}
}

\frame {
   \frametitle{Ограничение выставляемого интерфейса}

   Поскольку на Agda можно потребовать от аргументов функций свойств, не представимых
   в Haskell, то необходимо уметь запрещать давать прямой доступ к ним. Иначе,
   можно будет передать неправильный (с точки зрения Agda) аргумент, который пройдет
   проверку типов на стороне Haskell, и получить падение программы на этапе исполнения.
}

\frame {
   \frametitle{Существующие решения}

   \begin{description}
   \item[Coq 8.4pl3]
      Экстракция программ\footnote{P. Letouzey. \emph{A New Extraction for Coq}. 2002}.
      Генерируется код, из которого стираются все доказательства.
      Но это значит, что некоторые функции, требовавшие инварианты
      на этапе компиляции, теперь будут их требовать на этапе исполнения.
   \item[Agda 2.3.2.2]
      Компилятор MAlonzo\footnote{
      \url{http://thread.gmane.org/gmane.comp.lang.agda/62}}. Фокусируется
      на генерировании исполняемых файлов через трансляцию в Haskell.
      Генерирует имена вида буква+число, теряет всю информацию о типах (кроме
      арности функций).
   \item[Agda 2.3.4]
      Появилась возможность давать пользовательские имена функциям
      и генерировать для них разумные типы.
   \end{description}
}

\frame {
   \frametitle{Цели и задачи}

   \begin{block}{Цель работы}
      Разработать механизм для MAlonzo, генерирующий интерфейс на Haskell к коду
      на Agda, использование которого не позволит нарушить инварианты, поддерживаемые
      Agda.
   \end{block}

   Задачи:
   \begin{enumerate}
   \item Провести анализ методов трансляции и принципов генерации кода в компиляторе MAlonzo.
   \item Разработать механизм генерации интерфейса на Haskell к сгенерированному
         MAlonzo коду, не позволяющий нарушить инварианты, поддерживаемые Agda.
   \item Доказать корректность генерируемого интерфейса.
   \item Реализовать поддержку механизма экстракции в компиляторе MAlonzo и провести тестирование этой
         реализации.
   \end{enumerate}
}

\frame {
   \frametitle{Обзор реализации}

   \begin{itemize}
   \item Выставляемый интерфейс представляет собой обертки над кодом,
         сгенерированным MAlonzo, которые имеют типы, поддерживающие те
         же инварианты, что требует код на Agda.
   \item Язык Agda расширен прагмой
         \texttt{\{-\# EXPORT \(AgdaName\) \(HaskellName\) \#-\}}, которой
         передается имя из Agda и желаемое имя в Haskell. Если сущность
         \(AgdaName\) представима в Haskell и \(HaskellName\) --- разрешенное
         имя для этой сущности, то во время компиляции генерируется
         соответствующая обертка.
   \item Для модуля \(AgdaModuleName\) код интерфейса помещается в
         модуль \(MAlonzo.Export.AgdaModuleName\), чтобы отделить
         код, сгенерированный MAlonzo (находящийся в
         \(MAlonzo.Code.AgdaModuleName\)) от безопасного интерфейса.
   \end{itemize}
}

\frame {
   \frametitle{Подробности реализации}
   \framesubtitle{Тип данных и все его конструкторы выразимы}

   Можно полностью задать этот тип на Haskell и использовать его конструкторы
   для создания экземпляра и для сопоставления с образцом (pattern matching).
   \\[\baselineskip]
   Но для реализации нужно, чтобы тип в MAlonzo имел идентичную внутреннюю структуру
   типу интерфейса. Поэтому, необходимо подменять тип, генерируемый MAlonzo.
}

\frame {
   \frametitle{Подробности реализации}
   \framesubtitle{Тип данных выразим, но хотя бы один конструктор не выразим}

   Тип необходимо сделать абстрактным для внешнего кода. Ограничиваться генерированием
   только представимых конструкторов нельзя --- множество термов, имеющих данный тип
   будет отличаться между Agda и Haskell.
   \\[\baselineskip]
   Для реализации было принято решение делать \texttt{newtype} обертку над типом, генерируемым
   MAlonzo, что позволяет использовать \texttt{unsafeCoerce} для трансформации между ними.
   Такой же подход используется для простых типов.
}

\frame {
   \frametitle{Подробности реализации}
   \framesubtitle{Функции}

   Способ реализации параметрического полиморфизма отличается в Agda и в Haskell:
   Agda требует передавать параметр типа как аргумент функции, Haskell выводит его
   автоматически. MAlonzo при генерировании кода оставляет этот дополнительный аргумент,
   который всегда будет иметь значение \(Unit\).
   \\[\baselineskip]
   Если оставлять этот аргумент в генерируемом интерфейсе, то пользователю придется
   вручную передавать и пропускать \(Unit\). Поэтому, генерируются обертки, которые
   делают это автоматически.
}

\frame {\label{meta:slidecount}
   \frametitle{Выводы}

   Таким образом:
   \begin{enumerate}
   \item Экстракция кода из Agda в Haskell, сохраняющая семантику, возможна для широкого класса типов данных и функций Agda.
   \item Разработан способ генерировать безопасный интерфейс на Haskell к коду на Agda.
   \item Доказана его корректность.
   \end{enumerate}
}

\frame {
   \frametitle{Agda 2.3.4}

   \begin{itemize}
   \item Прагма \texttt{\{-\# COMPILED\_EXPORT \(AgdaName\) \(HaskellName\) \#-\}}.
   \item Подменяется имя функций (переименовывание типов данных не поддерживается), генерируемых MAlonzo.
   \item Функции требуют \(Unit\) на место параметров типов.
   \end{itemize}
}

\frame {
   \frametitle{Что дальше}

   \begin{itemize}
   \item Необходимо что-то сделать для экспортирования типов данных целиком: подменять код MAlonzo или
         генерировать биекции.
   \item В Agda есть способ симулировать классы типов из Haskell с помощью \textbf{record} --- можно
         попробовать генерировать соответствующие классы типов и их реализации.
   \item Существует множество расширений системы типов Haskell, которые позволяют в той или иной
         степени реализовывать зависимые типы --- их использование позволит выражать больше типов Agda
         в Haskell.
   \end{itemize}
}

\frame {
   \frametitle{\textbf{data} вместе с биекциями}

   \begin{align*}
   &\mathbf{newtype}\ List\ a,\quad\mathbf{data}\ T = In\ Int\ |\ Ch\ Char\\
   &trTtoT1 :: T \rightarrow T1,\quad trT1toT :: T1 \rightarrow T,\quad empty :: List\ a\\
   \\
   &add :: T \rightarrow List\ T \rightarrow List\ T\\
   &add = \lambda x\ xs.\ \texttt{unsafeCoerce}\ (d7\ (trTtoT1\ x)
      \ (\texttt{unsafeCoerce}\ xs))\\
   \\
   &head :: List\ a \rightarrow a\\
   &head = \lambda xs.\ {\color{red}\texttt{unsafeCoerce}}\ (d8
      \ (\texttt{unsafeCoerce}\ xs))\\
   \\
   &test = head\ (add\ (In\ 3)\ empty)
   \end{align*}

  Для корректной работы \(head\) обязан был вызвать \(trT1toT\) вместо
  \texttt{unsafeCoerce}.
}
