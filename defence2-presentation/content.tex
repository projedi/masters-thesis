\frame {
   \frametitle{Формальная верификация}

   \begin{itemize}
   \item Необходимо уметь убеждаться, что написанная программа решает
         поставленную задачу.
   \item Тестирование не может показать, что программа верна для всех случаев
         (если, конечно, нельзя сделать полный перебор).
   \item Формальная верификация позволяет сравнить программу с формальной
         математической моделью и доказать их эквивалентность на всех входных данных.
   \end{itemize}
}

\frame {
   \frametitle{Agda}

   \begin{itemize}
   \item Один из способов формально верифицировать --- строить формулы достаточно
         мощной логики над элементами программы и проверять их на этапе компиляции.
   \item Agda позволяет строить формулы на языке предикативной конструктивной логики.
         TODO: Definition of predicative logic?
   \end{itemize}
}

\frame {
   \frametitle{Использование верифицированного кода}

   Написание верифицированного алгоритма недостаточно --- необходимо еще
   использовать этот код из <<реальных>> приложений. Подходы:
   \begin{enumerate}
   \item Использовать Agda для написания приложений целиком.
         \begin{description}
         \item[\(+\)] Можно верифицировать больше кода.
         \item[\(-\)] Не Тьюринг-полный язык.
         \end{description}
   \item По коду на Agda генерировать код на другом языке
         \begin{description}
         \item[\(+\)] Удобнее писать <<реальный>> код.
         \item[\(-\)] Необходимо поддерживать корректность кода при трансляции.
         \end{description}
   \end{enumerate}

   Второй пункт называется <<экстракция программ>> и используется в системе Coq.
}

\frame {
   \frametitle{Цели и задачи}

   \begin{block}{Цель}
      По коду на Agda получить код на Haskell, которую можно использовать
      из программы на Haskell, не нарушая внутренние инварианты, поддерживаемые Agda.
   \end{block}

   Задачи: TODO
   \begin{enumerate}
   \item Придумать
   \item Реализовать
   \item Доказать
   \end{enumerate}
}

\frame {
   \frametitle{Выводы}

}
