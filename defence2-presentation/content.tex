\frame {
   \frametitle{Формальная верификация}

   \begin{itemize}
   \item Необходимо уметь убеждаться, что написанная программа решает
         поставленную задачу.
   \item Тестирование не может показать, что программа верна для всех случаев
         (если, конечно, нельзя сделать полный перебор).
   \item Формальная верификация позволяет сравнить программу с формальной
         математической моделью и доказать их эквивалентность на всех входных данных.
   \end{itemize}
}

\frame {
   \frametitle{Agda}

   \begin{itemize}
   \item Один из способов формально верифицировать --- строить формулы достаточно
         мощной логики над элементами программы и проверять их на этапе компиляции.
   \item Agda позволяет строить формулы на языке предикативной конструктивной логики.
         TODO: Definition of predicative logic?
   \end{itemize}
}

\frame {
   \frametitle{Использование верифицированного кода}
}

\frame {
   \frametitle{Цель}

}

\frame {
   \frametitle{Выводы}

}
